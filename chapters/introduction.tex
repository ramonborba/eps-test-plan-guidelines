%
% introduction.tex
%
% Copyright (C) 2024 by SpaceLab.
%
% EPS Test Plan Guidelines
%
% This work is licensed under the Creative Commons Attribution-ShareAlike 4.0
% International License. To view a copy of this license,
% visit http://creativecommons.org/licenses/by-sa/4.0/.
%

%
% \brief Introduction chapter.
%
% \author Ramon de Araujo Borba <ramonborba97@gmail.com>
%
% \version 0.1.0
%
% \date 2024/02/14
%

\chapter{Introduction} \label{ch:introduction}

This document contains a series of guidelines and recommendations regarding the preparation of test plans for CubeSat Electrical Power Systems.

This guidelines are based on the ECSS Standards ECSS-E-ST-10-02 \cite{ecss-e-st-10-02} and ECSS-E-ST-10-03 \cite{ecss-e-st-10-03} and are intended to be applicable to different EPS\nomenclature{\textbf{EPS}}{\textit{Electrical Power System}} topologies and architectures.

An important consideration regarding the ECSS standards is that they were designed with for large scale missions and satellites of large sizes, and many of the requirements defined are not feasible for small scale CubeSat missions, with limited resources and development time, and especially when applying for a single module.
The guidelines in this documents are an adaptation of the requirements defined in the standards, considering the scenario of a CubeSat mission.

Testing is considered part of the verification process outlined in the standards, this means that many of the concepts and considerations presented in this guidelines are derived from concepts related to this process.
With this in mind, a basic review of the verification process main concepts is advised before implementing the test plan.

This document will cover: test plan structure, test objectives, associated documentation, general requirements and provide a baseline test matrix.