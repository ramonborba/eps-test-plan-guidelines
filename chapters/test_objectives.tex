%
% introduction.tex
%
% Copyright (C) 2024 by SpaceLab.
%
% EPS Test Plan Guidelines
%
% This work is licensed under the Creative Commons Attribution-ShareAlike 4.0
% International License. To view a copy of this license,
% visit http://creativecommons.org/licenses/by-sa/4.0/.
%

%
% \brief Test Objectives chapter.
%
% \author Ramon de Araujo Borba <ramonborba97@gmail.com>
%
% \version 0.1.0
%
% \date 2024/02/14
%

\chapter{Test Objectives} \label{ch:test-objectives}

The ECSS standards \cite{ecss-e-st-10-03} define three main objectives for the testing process, qualification testing, acceptance testing and proto-flight testing.
This objectives are directly related to the verification stages and model philosophy concepts of the verification process.

For each of these objectives, different margins regarding test levels and durations are defined and, when applicable, should be adopted directly from the standards document \cite{ecss-e-st-10-03}.

Regarding the test margins and required environmental tests, the launch provider may have its specific requirements. These take priority over the requirements defined in the ECSS standards.

In addition to these main objectives, a test plan may fulfill other purposes, such as to provide data for comparing different modules, or evaluate performance in research scenarios, for example.
These additional objectives are not covered in the ECSS standards and are mainly relevant when the test plan is prepared for different purposes, outside of the verification process.

\section{Qualification Testing}

Qualification testing has the objective of verifying that the design of the item under test meets its applicable requirements, providing evidence that it performs in accordance to its specifications, considering the intended environment and qualification margins.

The qualification test plan is executed on dedicated qualification model units, manufactured specifically for the qualification tests.


\section{Acceptance Testing}

Acceptance testing has the objective of verifying that the item under test is in conformance with the verified design and is free from manufacturing defects or flaws, providing evidence that it performs in accordance to its specifications, considering the intended environment and acceptance margins.

The acceptance test plan is executed on every flight model unit.


\section{Proto-flight Testing}

Proto-flight testing combines the objectives of both qualification and acceptance testing. The proto-flight test plan may be executed on the first flight model unit and usually combines qualification test levels with acceptance test durations for its margins.